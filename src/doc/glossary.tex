\label{sec:glossary}

In this section, our terminology is described.

\subsection*{User}
Everyone, who logs into the application, is a \emph{user}. Each user has their own settings, authentication data etc., and can own multiple subtitle files, called \emph{Documents}.

\subsection*{Document}
A \emph{document} corresponds to a subtitle file, owned by a user. (This is not connected to files that are used for building the corpus initially.) The document contains information about the \emph{media source} of the subtitle file and a list of the \emph{chunks} of the subtitle file which either are waiting for translation or have already been translated -- in such case, it also contains their \emph{user translations}.

\subsection*{Media source}
A \emph{media source} denotes a movie or a TV show, which is the source of a document. It contains the name of the movie, its release year and a set of tags describing the genre of the movie.

\subsection*{Subtitle item}
A \emph{subtitle item} is a piece of text that has a time declaration in the subtitle file and is to be displayed at the given time when playing the movie.
% WTF:
% that has a time declaration in the video file and has a clear time information and is displayed in the video in a given time when playing.

\subsection*{Chunk}
A \emph{chunk} is a piece of text that is to be translated at once. As we will describe later, each subtitle item is split into one or more chunks.

Unfortunately, in the code itself, we also often use the term \emph{chunk} for the whole subtitle item. This is an error on our part, caused by not exactly defining all the terms beforehand and just using the term \emph{subtitle chunk} for everything.

\subsection*{Surface form}
Every chunk has a \emph{surface form} -- that means the text that is being translated. The surface form is without any non-textual information -- we store these in annotations.

\subsection*{Annotations}
Every chunk also has \emph{annotations} -- non-textual information that might be used at some point, but is not sent to the translation memory for querying. They mark positions of named entities, original position of newlines and dialog marks (``-'') -- none of these are thought to be relevant for translation.

\subsection*{Timed chunk}
A \emph{timed chunk} is a chunk that also carries timing information from its subtitle item and information about order in which it appears in the subtitle item.

Or, from the other point of view: a subtitle item is split into one or more timed chunks, which all have the same time declaration, but they are assigned 1-based indexes that denote their original order.

\subsection*{Translation suggestion}
A \emph{translation suggestion} is a piece of text that might be a translation of a given chunk, comming either from the Translation Memory or from Machine Translation. It typically also contains the matched text found in the Translation Memory, and a score.

\subsection*{User translation}
A \emph{user translation} is a piece of text produced by the user as the translation of a chunk. It may or may not be identical or similar to a translation suggestion.

\subsection*{Translation result}
A document that is being worked on consists of \emph{translation results}. Each translation result contains a timed chunk from the original subtitle file, and may also contain a list of translation suggestions or a user translation.

\subsection*{Page}
A \emph{page} in GUI (also often called ``screen'' by other authors) is a viewpoint in the GUI, having a distinct name, URL identifier, layout and function -- such as the Help Page or the Translation Workspace.

\subsection*{Dialog}
A \emph{Dialog} is an element smaller than a page in height and width, displayed on top of a page and disabling the user interaction with the page. Similarly to a page, it also has a distinct layout and function. It typically contains text boxes to fill in or edit some values, and buttons to submit the values or to close the dialog.
