Since we wanted the translation memory to be functional from the very beginning, we needed to build some parallel corpus -- best from some repository of subtitles. 

We first get data from OpenSubtitles. Once we have the "raw" subtitle data, we build a parallel corpus from that data to use for translation memory. 

Since there are more strategies for building the corpus from the "raw" data, we also want to somehow measure the quality of the resulting corpus (either by measuring the alignment itself or the resulting corpus).

In this chapter, we discuss these steps, together with more detailed description of the data we got. We will also discuss various strategies for measuring the quality of the corpus.


\section{Retrieving data}

There are plenty of subtitle files in many languages available at the Internet these days, which can be easily downloaded.
However, it is problematic to download from most of the servers in bigger batches (thanks to anti-spam protection and so on); also, we wanted to avoid any copyright problems we would meet by using random subtitles from the internet. For this purpose, we asked the administrator of the biggest server providing the subtitles \emph{opensubtitles.org} for the data.

As a result of this we received all subtitle files in Czech and English from \emph{opensubtitles.org} with following licence condition (in Slovak):

\begin{quote}
\begin{verbatim}
Titulky mozem poskytnut, s tym ze:

- nebudu sa dalej sirit
- vsade, kde je to mozne a suvisi to s projektom, bude uvedena linka na
    www.opensubtitles.org (stranka programu, dokumentacia, program...)

Co sa tyka autorskych prav, tak neviem presne ako to je, ale myslim,
ze to je +- ok :)
\end{verbatim}
\end{quote}

\noindent with following English translation:

\begin{quote}
\begin{verbatim}
We can provide the subtitle files under following conditions:

- they won't be provided any further
- a link to www.opensubtitles.org will be placed whenever it's possible 
   (web page of the program, documentation, program itself...)

Considering the copyright law, I am not really sure how it is, 
but I think it's ok :)
\end{verbatim}
\end{quote}


\noindent We decided that this license condition is acceptable for our purposes. 

As a matter of fact, the users of \emph{opensubtitles.org} agree with a statement where they declare they are holders of all rights to the content they post to the server, and provide the subtitle files as their own intellectual property for public use. Based on this and the license, we think there are no more copyright issues.% Based on this statement we trust the users they really did what they declared.

%From this data we will create a parallel corpus of ...

\section{The initial data properties}

We received 3,076 MB of data in 139,538 files with a database index dump which did not exactly match the received content. After removing files, which were not in the index, and removing index data, that didn't have the files, we have 39,712 Czech subtitle files and 97,991 English files of 15,881 movies or TV shows episodes -- 3,032 MB of data. (Each file was individually zipped with \texttt{gzip}, so unzipped version would be probably bigger.)

Some subtitles are divide into more files, anyway 81\% of subtitles are in one piece.
There is also just 1.7\% movies having subtitles only in multiple files. 
We assumed that those split into more parts are probably just split version the complete ones. 
So, to keep the chunk alignment simple, we deleted the split subtitle files.

This caused that some movies lost its translation -- 64
of Czech and 218 of English. In total we will lost 3,5\% of movies. After this steps, the amount of data was 2,556 MB.

Because OpenSubtitles internally uses IMDB for adding information about movies and TV shows to subtitles, it was not hard to align subtitles of same movies and TV show episodes together, since all movie names used the same format (name and year), and the TV episodes were all correctly marked.

However, this also caused one curious issue.
While looking more carefully at the data, we found there were 228 Czech subtitles files, 814 files in total having one particular movie ID and containing various content. This movie was Carmentica, a 21 seconds long silent film from 1884. This happened due to a server error at \emph{opensubtitles.org}, because the movie has ID {\tt tt0000001} at IMDB.

After doing all mentioned filtering we had 2,543 MB in 110,312 subtitle files (32,705 Czech, 77,607 English) of 15,552 movies / TV shows' episodes.

\todo{!!!???do something with it??}Our use of the word \emph{chunk} is not consistent in this document, however, in this part, I will call "chunk" the part of data, that has time mark and is displayed on the screen at one time. With that definition, there is \todo{find out!!}XXX chunks in the data, in average, XXX chunks on one file.

\section{Subtitles and sentences}

By observing several subtitle files, we can see some patterns. 

\begin{enumerate}
    \item Usually, dialogues are marked with dash in the beginning. Example (from english subtitles from movie \emph{Jurassic Park III}):
    
        \texttt{7 \\
        00:01:40,224 --> 00:01:42,762 \\
        -Ready, amigo? \\
        -Ready!}
        
    \item Sometimes, there is a sentence spread through more "screens" (or what we call \emph{chunks})
    
    \texttt{62
    00:05:35,327 --> 00:05:36,092\\
    Ellie,\\
\\
    63\\
    00:05:36,127 --> 00:05:37,967\\
    all the theories are about\\
    raptors intelligence,\\
\\
    64\\
    00:05:38,002 --> 00:05:39,567\\
    what are they capable of?}
    
    \item Sometimes, there are more sentences in one chunk
    
    \texttt{43\\
    00:04:43,047 --> 00:04:46,607\\
    This is Hylaeosaurus.\\
    That's the dinosaur man.}
    
\end{enumerate}

As you can see, the sentences and chunks don't exactly match each other.

\section{Aligning the subtitles}

While we had enough metadata to align subtitles of the same movie together, there were still more than two subtitle files for each of the movies, so we needed to chose the best pair. Also, on these, we needed to align the actual chunk pairs.

We used two different approaches for this. All of them need to solve these three issues:
\begin{itemize}
    \item file to file alignment -- in each movie find \emph{one} best pair
    \item filter the "good" movies -- filter the movies, whose best pairs are well alignable
    \item chunk to chunk alignment -- if we already have pairs of files, chose the best  
\end{itemize}

All those are implemented as abstract classes in \texttt{cz.filmtit.dataimport.alignment.model} and their two implementations (described below) in \texttt{cz.filmtit.dataimport.alignment.aligners}.

\subsection{File to file alignment}
In general, there is 2.1 Czech subtitle files and 5 English subtitle file. However, the numbers are not evenly distributed and very popular movies can have tens of subtitles.

What is, however, helping us in determining alignment is the fact, that Czech subtitles are usually taken from existing subtitles in English and directly translated, therefore the time marks should be similar. It is interesting to note that in professional translation of movie subtitles, this is not the case -- as was noted to us by professional movie subtitle translators, professional subtitles are translated directly from the script and then recut to not be too long or too short, which is usually not a concern for fan made translations. 

On the other hand, usually, people, who download movies (and, therefore, subtitles) online, download them from sources of questionable legality, which usually have more versions of the same file (some are so-called "cam-rips", some are copied from DVDs or blu-rays). Similarly with TV episodes -- depending on the source, the timings are slightly different. Also, in some cases, the subtitle translations are directly copied from DVD subtitles (which is a case of professional translation with different timing).

Also, sometimes, subtitles have absolutely wrong movie tag.

For illustration here, I randomly selected a movie from the database and show sample of its subtitles. The randomly selected movie is \emph{Legends of the Fall} from 1994 - 5 lines from beginning, 3 lines from the middle (374 to 376, if possible), 3 lines from the end.
 
\todo{better formating}

\newenvironment{subexam}{\begin{boxedminipage}[b]{0.4\textwidth}
\footnotesize
\begin{alltt}
}
{
\end{alltt}
\end{boxedminipage}}



\begin{subexam}
1
00:00:00,066 --> 00:00:02,375
Titulky preložené do češtiny by BiGfOoT.
----------------------------------------
Tento disk DVD (Digital Versatile Disc)
je určen pouze pro domácí užití.
Veškerá práva k obsahové náplni včetně
zvukového záznamu přísluší vlastníku
autorského práva.

2
00:00:02,746 --> 00:00:05,055
Neautorizované rozmnožování,
úpravy, projekce pro jiné než domácí
účely, pronájem, výměna, pújčování a
jakákoli forma přenosu tohoto disku
DVD nebo jeho částí jsou zakázány.
Porušování práv vlastníka autorského
práva bude stíháno podle platných
právních předpisú.

3
00:00:32,666 --> 00:00:38,059
LEGENDA O VÁŠNI

4
00:00:46,426 --> 00:00:51,454
Někteří lidě slyší svúj
vnitřní hlas nadmíru jasně.

5
00:00:51,626 --> 00:00:54,663
A řídí se jím po celý život.

374
00:56:41,186 --> 00:56:44,781
Řekni to ještě jednou
a přestaneme být bratry.

375
00:56:46,506 --> 00:56:49,862
Někdy!

376
00:56:52,146 --> 00:56:56,503
- S tebou nebude šťastna.
- Uvidíme.

755
02:01:42,266 --> 02:01:47,101
... někde mezi tímto světem
a tím druhým.

756
02:02:23,266 --> 02:02:26,178
Byla to dobrá smrt.

757
02:07:18,306 --> 02:07:22,697
České titulky - BiGfOoT.
\end{subexam}
\hspace{0.5cm}
\begin{subexam}
1
00:00:00,066 --> 00:00:02,375
Titulky preložené do češtiny by BiGfOoT.
----------------------------------------
Tento disk DVD (Digital Versatile Disc)
je určen pouze pro domácí užití.
Veškerá práva k obsahové náplni včetně
zvukového záznamu přísluší vlastníku
autorského práva.

2
00:00:02,746 --> 00:00:05,055
Neautorizované rozmnožování,
úpravy, projekce pro jiné než domácí
účely, pronájem, výměna, pújčování a
jakákoli forma přenosu tohoto disku
DVD nebo jeho částí jsou zakázány.
Porušování práv vlastníka autorského
práva bude stíháno podle platných
právních předpisú.

3
00:00:32,666 --> 00:00:38,059
LEGENDA O VÁŠNI

4
00:00:46,426 --> 00:00:51,454
Někteří lidě slyší svúj
vnitřní hlas nadmíru jasně.

5
00:00:51,626 --> 00:00:54,663
A řídí se jím po celý život.

374
00:56:41,186 --> 00:56:44,781
Řekni to ještě jednou
a přestaneme být bratry.

375
00:56:46,506 --> 00:56:49,862
Někdy!

376
00:56:52,146 --> 00:56:56,503
- S tebou nebude šťastna.
- Uvidíme.

755
02:01:42,266 --> 02:01:47,101
... někde mezi tímto světem
a tím druhým.

756
02:02:23,266 --> 02:02:26,178
Byla to dobrá smrt.

757
02:07:18,306 --> 02:07:22,697
České titulky - BiGfOoT.
\end{subexam}


\begin{subexam}

1
00:00:46,577 --> 00:00:51,605
Somepeople heartheir own
innervoices with greatcleamess.

2
00:00:51,777 --> 00:00:54,814
And they live bywhatthey hear.

3
00:00:54,977 --> 00:01:00,973
Such people become crazy,
orthey become legends ...

4
00:01:07,137 --> 00:01:11,813
Tristan Ludlowwas bom
in the moon ofthe falling leaves.

5
00:01:11,977 --> 00:01:14,332
ltwas a terrible winter.

374
00:56:41,337 --> 00:56:44,932
You saythat again
and we're not brothers.

375
00:56:46,657 --> 00:56:50,013
Once!

376
00:56:52,297 --> 00:56:56,654
- You know you can't make her happy.
- l'm gonna try.

756
02:01:38,937 --> 02:01:42,247
He hadalways lived
in the borderland, anyway.

757
02:01:42,417 --> 02:01:47,252
Somewhere between this world
and the other.

758
02:02:23,417 --> 02:02:26,329
lt was a good death.
\end{subexam}
\hspace{0.5cm}
\begin{subexam}
1
00:00:58,000 --> 00:01:00,696
- I'm here!
- Where? I can't see.

2
00:01:02,704 --> 00:01:04,365
I can't move!

3
00:01:08,577 --> 00:01:10,238
I'm coming!

4
00:01:35,671 --> 00:01:38,037
I got you now.

5
00:01:40,008 --> 00:01:42,272
You're doing good.

315
00:44:59,438 --> 00:45:02,032
were married several years ago.

316
00:45:05,278 --> 00:45:07,542
Your brother's a congressman now.

317
00:45:09,282 --> 00:45:11,716
They have a big, new place
in Helena.
\end{subexam}


\begin{subexam}
1
00:00:45,869 --> 00:00:49,186
<i>Some people hear 
their own inner voices...</i>

2
00:00:49,261 --> 00:00:50,920
<i>with great clearness...</i>

3
00:00:50,989 --> 00:00:54,087
<i>and they live by what they hear.</i>

4
00:00:54,157 --> 00:00:57,125
<i>Such people become crazy...</i>

5
00:00:57,198 --> 00:00:59,369
<i>or they become legends.</i>

374
00:34:22,150 --> 00:34:23,710
Charge!

375
00:34:37,734 --> 00:34:39,806
Goddamn it!

376
00:34:40,838 --> 00:34:42,693
- Where are you hit?
- It's just a scratch.

983
02:01:38,444 --> 02:01:41,891
<i>He had always lived
in the borderland anyway:</i>

984
02:01:41,964 --> 02:01:45,576
<i>somewhere between this world
and the other.</i>

985
02:02:23,086 --> 02:02:25,093
<i>It was a good death.</i>
\end{subexam}
\hspace{0.5cm}
\begin{subexam}
1
00:00:44,411 --> 00:00:47,869
Some people hear their own
inner voices

2
00:00:47,948 --> 00:00:49,677
with great clearness

3
00:00:49,750 --> 00:00:52,981
and they live
by what they hear

4
00:00:53,053 --> 00:00:56,147
Such people become crazy

5
00:00:56,223 --> 00:00:58,487
or they become legend

374
00:35:27,326 --> 00:35:30,921
Is that wrong to want
to distinguish myself gloriously

375
00:35:30,996 --> 00:35:33,226
in combat as my father did?

376
00:35:33,298 --> 00:35:36,290
Tristan and Alfred
watch over me so carefully

1000
02:06:50,103 --> 02:06:53,869
Somewhere between this world
and the other

1001
02:07:32,979 --> 02:07:35,072
It was a good death

1002
02:12:47,200 --> 02:12:49,072
\{\{\{the end\}\}\}
\end{subexam}


\begin{subexam}
1
00:00:43,411 --> 00:00:46,869
Some people hear their own
inner voices

2
00:00:46,948 --> 00:00:48,677
with great clearness

3
00:00:48,750 --> 00:00:51,981
and they live
by what they hear

4
00:00:52,053 --> 00:00:55,147
Such people become crazy

5
00:00:55,223 --> 00:00:57,487
or they become legend

374
00:35:26,326 --> 00:35:29,921
Is that wrong to want
to distinguish myself gloriously

375
00:35:29,996 --> 00:35:32,226
in combat as my father did?

376
00:35:32,298 --> 00:35:35,290
Tristan and Alfred
watch over me so carefully

1000
02:06:49,103 --> 02:06:52,869
Somewhere between this world
and the other

1001
02:07:31,979 --> 02:07:34,072
It was a good death

1002
02:12:46,200 --> 02:12:48,072
\{\{\{the end\}\}\}
\end{subexam}
\hspace{0.5cm}
\begin{subexam}
1
00:00:46,418 --> 00:00:51,446
Some people hear their own
inner voices with great clearness.

2
00:00:51,618 --> 00:00:54,655
And they live by what they hear.

3
00:00:54,818 --> 00:01:00,814
Such people become crazy,
or they become legends ...

4
00:01:06,978 --> 00:01:11,654
Ôristan Ludlow was born
in the moon of the falling leaves.

5
00:01:11,818 --> 00:01:14,173
It was a terrible winter.

374
00:56:58,898 --> 00:57:02,208
You will fail.

375
00:57:14,298 --> 00:57:17,529
I'm going to be leaving today.

376
00:57:26,458 --> 00:57:30,087
l do wish you both all the best.

752
02:01:42,258 --> 02:01:47,093
Somewhere between this world
and the other.

753
02:02:23,258 --> 02:02:26,170
It was a good death.

754
02:07:18,298 --> 02:07:22,689
English subtitles - lFT
\end{subexam}

\begin{subexam}
1
00:00:45,977 --> 00:00:51,005
Some people hear their own
inner voices with great cleamess.

2
00:00:51,177 --> 00:00:54,214
And they live by what they hear.

3
00:00:54,377 --> 00:01:00,373
Such people become crazy,
or they become legends ...

4
00:01:06,537 --> 00:01:11,213
Tristan Ludlow was bom
in the moon of the falling leaves.

5
00:01:11,377 --> 00:01:13,732
lt was a terrible winter.

374
00:56:40,737 --> 00:56:44,332
You saythat again
and we're not brothers.

375
00:56:46,057 --> 00:56:49,413
Once!

376
00:56:51,697 --> 00:56:56,054
- You know you can't make her happy.
- l'm gonna try.

756
02:01:38,337 --> 02:01:41,647
He had always lived
in the borderland, anyway.

757
02:01:41,817 --> 02:01:46,652
Somewhere between this world
and the other.

758
02:02:22,817 --> 02:02:25,729
lt was a good death.
\end{subexam}
\hspace{0.5cm}
\begin{subexam}
1
00:00:43,785 --> 00:00:47,243
Some people hear their own
inner voices...

2
00:00:47,322 --> 00:00:49,051
with great clearness...

3
00:00:49,123 --> 00:00:52,354
and they live
by what they hear.

4
00:00:52,427 --> 00:00:55,521
Such people become crazy...

5
00:00:55,597 --> 00:00:57,861
or they become legends.

374
00:35:35,741 --> 00:35:37,470
I may never get the opportunity.

375
00:35:46,085 --> 00:35:47,712
Charge!

376
00:36:02,334 --> 00:36:04,495
Goddamn it!

987
02:06:45,772 --> 02:06:49,367
He had always lived
in the borderland anyway:

988
02:06:49,443 --> 02:06:53,209
somewhere between this world
and the other.

989
02:07:32,319 --> 02:07:34,412
It was a good death.
\end{subexam}

As you can see, the two czech subtitles are exact duplicates. Also, there is one English subtitle that is totally wrong. Except this one, though, most of the English files are correct "translation" of the Czech one. You can also see neither one is a "perfect" match, if you look either at numbers or at time marks; but -- judging by the \emph{number} of the last translation -- some are from similar source (those with around 750 chunks), while those with significantly more chunks are probably from different source.

\todo{This is where I ended up, I have to go to sleep, gotta finish the rest later}

\subsubsection*{Editing distance}

While comparing two mostly identical subtitle files, there may appear some issues which cause that the timing of the files is not identical. There exist some cases where one chunk is split into more or two are merged into one. In both of this cases, two time declarations in the file remain the same and two time declarations are added or deleted. There are also a lot of subtitles for deaf people where from time to time some additional subtitle appears. 

From that we concluded that the best measure how subtitle files matches would be the editing distance of their time declarations since the cases mentioned above contributes relatively little to the score in contrast to some more significant mismatches. 
By that we mean taking all the time information (both the starts and the ends) as vectors, and then counting editing distance of those two vectors by dynamic algorithm.

As one of the papers (CITE!!!) proposed we used 0.6\,s as a tolerance for equality, not to be confused by slight differences in timing. Because the computation on whole files would be really time consuming, we limited it just for the first 100 time declarations in the files.

%when I rewrote it to scala it was actually much faster then in perl. Maybe we COULD do the whole files now :-)

The results were following: from 15,552 movies there was 22.2\,\%
with perfect matches and 3.1\,\% of total mismatches. The scores for partial matches are captured in table \ref{opensubtitles:matchTable}.

\begin{table}[h]

\begin{center}
\begin{tabular}{|c|c|}
\hline
amount of films & measure of match\\ \hline
22,2 \% & $= 100 \%$ match \\
45.7 \% & $\ge 90 \%$ match \\ 
56.2 \% & $\ge 80 \%$ match \\ 
63.0 \% & $\ge 70 \%$ match \\
69.2 \% & $\ge 60 \%$ match \\ \hline
\end{tabular}
\end{center}

\caption{Table capturing for how many movies there exist a matching pair of subtitle files with given measure of matching}\label{opensubtitles:matchTable}
\end{table}

%what does the "measure of match" mean in this context? I don't get it.

After looking at some randomly selected files we have decided to use just movies for which we have a pair of files with at least 70 \% match. Surprisingly, some worse scored pairs have quite high-quality translations, because they contain a lot of joined and split chunks, but generally, based on ran the quality of translation decreased with the match score.

On this files we run an aligning algorithm based just on the timing. The chunks were aligned if both the time of their start and time of their end differ less than by 0.6\ s. This gave us 884 MB of parallel data which consists of 13,636,022 chunks. In this we have 5,669,837 unique chunks, from which 3.7 \% appears more than once. On the other hand, chunks appearing more than once make 57.6 \% of the whole corpus.
