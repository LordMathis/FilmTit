\chapter{Introduction}

The translation of the movie and TV shows subtitles is an activity that a lot of people do in their free time. There exist many sophisticated tools to assist the translators with the subtitle translation in the commercial wold, anyway a complex tool for a fan based translation using the latest state-of-art methods is still missing. Moreover, it is the noncommercial and community character of the fan translation which causes there is huge (and still growing) amount of data available for the for development of such tools.

The subtitles also seem to be a very good source of the parallel data. They are also often used for creation of the parallel corpora for the statistical machine translation (e.g. the CzEng corpus, http://ufal.mff.cuni.cz/czeng/). In the film industry there also exist attempts to do an automatic translation of the subtitles just with little post-editing based on the existing subtitles data.

The translation memories are in these day the most common tool used by the translators. The memories are usually very domain specific because of the efforts to make the computation feasible on the translators' PC and not to  bring any confusion to the used terminology. In the case of the movie subtitles there is no such terminological danger, moreover we believe that the similarity of the movie scripts will cause that more data will lead to better results.

The goal of our project is to create an application that will help the amateur translators of the movie and TV shows subtitles with their effort. The core of the application is a big translation memory, which will be gradually extended and improved by using it and will exist only in one publicly available instance. We focus on the English-Czech language pair, anyway the chosen solution is language independent.

Except the original purpose it will also possible to use the collected data for creation of a parallel corpus for a statistical machine translation system and for a research of the language used in the subtitles.
