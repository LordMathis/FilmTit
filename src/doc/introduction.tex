
The translation of subtitles of movie and TV shows is an activity that \todo{"a lot of people" sound too general IMO} a lot of people do in their free time.
There are many sophisticated commercial tools available for assisting translators with the task, however a complex tool for a fan based translation using the latest state-of-art methods is still missing. 
Moreover, because of the noncommercial and community character of the fan translation, there is a huge (and still growing) amount of \todo{I don't get what you mean by "data" in this context.  The subs themselves? Why do we use "also" in th next section then?}data available for the for development of such tools.

The subtitles themselves also seem to be a very good source of the parallel data. 
For example, they are \todo{"often" seems probably too general}often used for creation of the parallel corpora for the statistical machine translation (e.g. the CzEng corpus, http://ufal.mff.cuni.cz/czeng/).
In the film industry, there are attempts to do an automatic translation of the subtitles just with little post-editing based on the existing subtitles data.\todo{Any source on that?}

Translation memories are in these days \todo{Any source on that?}the most common tool used by translators.
The memories are usually very domain specific because of the efforts to make the computation feasible on the translators' PC and avoid any confusion in the used terminology. 
In the case of movie subtitles there is no such terminological danger, moreover we believe that the similarity of the movie scripts will cause more data  leading to better results.

The goal of our project is to create an application that will help the amateur translators of the movie and TV shows subtitles with their effort.
The core of the application is a big translation memory, which will be gradually extended and improved by users' translations and will exist only in one publicly available instance. We focus on the English-Czech language pair, but the chosen solution is language independent. \todo{"solution" means what? should use "our application" I think. Also maybe "should" instead of "is" , we never tried it :))}


Even with our original purpose being different, it will also be possible to use the collected data for creation of a parallel corpus for any statistical machine translation system and for a research on the language used in the subtitles in general.
