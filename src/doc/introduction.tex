On the internet, thanks to easy sharing of video files, there is quite a big community of movie and TV series fans that spend a significant amount of their free time by translating the subtitles of the foreign movies to their native languages. 

There are many sophisticated commercial tools available, that assist professional translators with their translation tasks, and software solutions used by big translation agencies, however, a complex tool for a fan based translation using the latest state-of-the-art technique is still missing. In contrast to the area of commercial translation, where each translation agency keeps already translated texts as their valuable property, which can make them later much more effective, because of the noncommercial and community character of the fan translations huge amount of translated subtitles is publicly available. Anyway, there has not been a tool for subtitle translation taking advantage of the public availability.

The subtitles themselves seem to be a very good source of the parallel data for linguistic research. They are used for creation of the parallel corpora for the statistical machine translation (e.g. the CzEng corpus \footnote{http://ufal.mff.cuni.cz/czeng/} used for training the statistical machine translation tools at ÚFAL). The film industry is also aware of this fact, there has been some attempts to do an automatic translation of the subtitles just with little post-editing based on the existing subtitles data.\todo{... I'll find that paper (Jindřich)}

Translation memories (discussed in more detail in section \ref{sec:translation_memories}) are in these days the most common tool used by translators. The memories are usually very domain specific because of the efforts to make the computation feasible on the translators' PC and to avoid any confusion in the used terminology. In the case of movie subtitles there is no such terminological danger. Moreover, we believe that the similarity of the movie scripts will cause more data leading to better results. These arguments make us to believe that creating a translation memory based tool focusing on the movie subtitle translation could be both useful and may gain also some success among user.

As it was indicated in the previous paragraphs, the goal of our project is to create an application that will help the amateur translators of the movies and TV shows subtitles with their effort. The core of the application is a big translation memory, which will be gradually extended and improved by users' translations. The translation memory exists only in one publicly available instance which is on the server as is therefore share by all the. We focus on the English-Czech language pair, but the application itself is language independent. Anyway, training data for statistical natural language processing tools are needed for anyone who would like to run the application for another language pair.

Although our original goal was to build a translation memory only, we found out that the collected data can be used in building a parallel corpus for training a statistical machine translation system. We built such a system, using Moses machine translation engine and added it into our project as another option for a translator. 

The collected data can be also used for a research on the language used in the subtitles in general.

The following chapter contain a detailed specification of the application. It is based on the specification which has been submitted to the committee at the time the project was assigned to us, but contains also technical issues which have not been clear at the project started and additional functionality we have implemented. 

The next chapter contain a general introduction to the area of producing subtitles and the computer assisted human translation. The chapter after it brings general information about implementing the project including usage of external libraries \todo{don't forget to place it there}. Following four chapters contains detailed description of the particular project modules.

Chapter \ref{chap:technical_manual} contains a manual for a server administrator how to install the server part of the application. Chapter \ref{chap:users_manual} contains the manual for the web application users. The last chapter is devoted to evaluation of development process of the project.