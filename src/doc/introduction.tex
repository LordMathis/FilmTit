Thanks to the effortless sharing of video files, there is a relatively big community of movie and TV series fans on the Internet who spend a significant amount of their free time translating the subtitles of foreign movies to their native languages.


There are many sophisticated commercial tools available that assist professional translators with their translation tasks, as well as software solutions used by big translation agencies; however, a complex tool for fan-based translation using  state-of-the-art techniques is still missing.

In contrast to the area of commercial translation, where each translation agency keeps previously translated texts as proprietary knowledge, fan translations are based on a non-commercial community approach that creates huge amounts of translated subtitles that are publicly available. There has not been a tool for subtitle translation taking advantage of the public availability of this data.

The subtitles themselves are an ideal source of parallel data for linguistic research. They are used for the creation of  parallel corpora for statistical machine translation (e.g.\ the CzEng corpus\footnote{\url{http://ufal.mff.cuni.cz/czeng/}} used for training the statistical machine translation tools at ÚFAL). The film industry is also aware of this fact and there have been some attempts to do an automatic translation of  subtitles  with minimal post-editing based on existing subtitles data.\footnote{Marian Flanagan (2009): \emph{Using Example-Based Machine Translation to translate DVD Subtitles.} Proceedings of the 3rd Workshop on Example Based Machine Translation, p. 85–92}

Translation memories (discussed in more detail in Section~\ref{sec:translation_memories}) are in these days the most common tool used by translators. The memories are usually very domain specific to make the computation feasible on the translators' PC and to avoid any confusion in terminology. In the case of movie subtitles, there is no such terminological danger. Moreover, we believe that the similarity of the movie scripts will produce more data leading to better results. These arguments make us believe that creating a translation memory based tool focusing on movie subtitle translation would be both useful and may also gain success among users.

As it was indicated in the previous paragraphs, the goal of our project is to create an application that will help amateur translators of movies and TV show subtitles with their efforts. The core of the application is a big translation memory, which will be gradually extended and improved by users' translations. The translation memory exists only in one publicly available instance, which is on the server, and is therefore shared by all users. We focus on the English-Czech language pair, but the application itself is language independent. Training data for statistical natural language processing tools are needed for anyone who would like to run the application for another language pair.

Although our original goal was to build a translation memory only, we found that the collected data can be used in building a parallel corpus for the training of a statistical machine translation system. We built such a system, using the Moses machine translation engine and added it into our project as another option for translators; we describe the system below.

The collected data can be also used for research on the language used in the subtitles in general.

The following chapter contains a general introduction to the area of producing subtitles and computer assisted human translation. The next chapter provides general information about implementing the project, including usage of external libraries. The following four chapters contain detailed description of the particular project modules.

After the project documentation, a part containing manuals follows. Chapter~\ref{chap:technical_manual} contains a manual for a server administrator on how to install the server part of the application. Chapter~\ref{chap:users_manual} contains the manual for web application users. The last part is devoted to the evaluation of the development process of the project.