
\section{Application}

The application has two parts -- the server part consisting of  two components
\begin{smallenumerate}
\item the translation memory and a server application mirroring the users activity in the client
\item the web application client providing the user interface in the web browser
\end{smallenumerate}

The user interface provides the user a practical and intuitive environment for the translation and  communicates with the translation memory.%technically speaking, we are communicating with userspace, not the TM
From the user's view, the main functionality is parsing the subtitle into individual chunks %shouldn't  we define what a "chunk" is? :)
and providing the translation suggestions to each of them,  with the possibility to play the corresponding pieces of the movie stored on the translator's computer with the subtitles either in the source or target language. 
The interface provides complex tool for the subtitle editing including the possibility to adjust the timing, add new chunks or delete some. %not right now it doesn't
In addition to the suggestions from other movie subtitles, a machine translated sentences are provided from the Moses system trained on the movie subtitle data.

The server part queries the translation memory for similar sentences which have occurred in the subtitles before and has been already translated. Different method of retrieving the similar chunks are used ... For evaluating how much the ...

The translation memory will accept the queries of the client application for the individual subtitle chunks, will search for simillar already used texts in the database and send several best ones back. For evaluation we will try not only the usual metrics, but also the feedback from users, who will have he ability to rate the imports of the translation memory. We will also try the possibilit of using some meta information like the agreement %????is agreement the right word?
of the TV show season, the agreement of genre etc. At the same time, the translation memory will accept all the translations made by the users and save it back to the database, so the quality of translation memory data increases over time.


The translation memory itself is implemented in Scala using the Postgre SQL databese. The server side is implemented in Java, the client is developed in Java, using Google Web Toolkit. (Google Web Toolkit -- or GWT -- is translating the Java code into a JavaScript code).

The communication between the server and client is using GWT's remote procedure calls (RPC), which are themselves implemented as asynchronous POST requests in GWT.
%more about GWT/Scala here? I don't think so

\section{Data}

Processing the movie subtitles is in many ways different from processing a general text. We believe that those differences will make a lot of tasks much easier, e.g. the sentence alignment.

%Věříme, že většina odlišností bude představovat spíše výhody (například omezená doména nebo velké množství dat využitelných pro počáteční naplnění překladové paměti), jistě tomu tak ale nebude ve všech případech.
%Mam tohle rekladat/should I rewrite that?

The subtitles in general consist of small, few-lined chunks of text, that are described by the time they should appear in the movie. It is important to note that the division of those chunks is not the divion by sentences; sentences are often divided in half and vice versa -- the chunks are often consisting from more sentences.    

Therefore, it is necessary to set a suitable basic unit of text which will be stored in the translation memory. In the statistical machine translation context, it is usual to use alignment on the sentence level, but the subtitle chunks often split the sentence into more pieces and on the other hand often contains more than one sentence. 
We  are trying to take this as an advantage and split the data into chunks and then, split the chunks on  sentence boundaries whenever it is possible. Because of the limited size of the subtitle chunks, we hope that using this sentence and sub-sentece level splitting will lead to enough repetitivness.

%V první řadě je nutné stanovit vhodnou základní jednotku textu, která bude celistvě ukládána do překladové paměti. Je obvyklé používat párování na úrovni vět, nicméně jednotlivé úseky titulků často dělí věty na více částí a/nebo sdružují více vět dohromady, k čemuž lze přistupovat jako k šumu, který je třeba odstranit, ale i jako k informaci navíc, kterou lze využít.

We hope it will be possible to get better results by pre-processing of named entities in the data (names, numbers etc.) and eventually their separate translation (in which case the data in translation memory could be used for building a language model).

%Pravděpodobně bude možné zlepšit výsledky preprocessingem pojmenovaných entit (jména, čísla apod.), případně i jejich odděleným překladem (přičemž data obsažená v překladové paměti je v tomto případě možné využít pro vybudování jazykového modelu).
With movie subtitles we can also use metadata from IMDB (International Movie Database) from their API or from TMDb (The Movie Database), a similar open website. We will try to use at least the genre of the movie (and in the translation suggestions to prefer the ones with similar genres). We can also use information like names of the characters (the possibility of connecting with preprocessing of named entities is obvious here)

%Filmové titulky také umožňují využití mnoha metadat, která lze získat pomocí API IMDb.com (International Movie Database) nebo z její otevřené obdoby TMDb.org (The Movie Database). Pokusíme se využít alespoň žánr filmu (a v návrzích překladu upřednostnit ty z žánrově podobných filmů), případně i další informace, jako jsou jména postav ve filmu (zde se samozřejmě nabízí propojení s preprocessingem pojmenovaných entit).

As an initial source for translation database we will use movie and TV series subtitles publicly available on the internet. For example on the server opensubtitles.org, which is often used for similar tasks, there is at the moment more than 65 thousands of movie subtitles in Czech. Needed alignment of subtitles is simpler than for general texts (mainly thanks to specific structure of subtitles, as described above) and there are effective methods for that.

%Jako zdroj dat pro počáteční naplnění překladové paměti budou využity filmové a seriálové titulky volně dostupné na internetu. Například na serveru    opensubtitles.org, který je často používán pro podobné účely, je v současné chvíli k dispozici více než 65 000 filmových titulků v češtině. Potřebné spárování titulků je navíc snazší než párování obecných textů (zejména díky specifické struktuře titulků) a existují pro něj efektivní metody
