\section{What is a Translation Memory?}

A translation memory is a tool for a machine assisted human translation. The core idea behind translation memories is an assumption that sentences that are similar in the source language will have probably a similar translation in the target language. If the tool is able to provide the translator a translation of a similar sentence, there is a big chance the translator can just a little bit edit the provided sentence and get the translation he wants.

The translation memories are widely used in the translation industry, mostly while translating technical documentation and localization of software when the translators usually take advantage from that the new version of manuals of software does not differ much from the previous version. Usually there is an effort to keep the translation memories as clean as possible -- to contain only relevant in domain sentences. The reasons to do that are to keep the database as small as possible not to make the search too slow and not spoil the terminology that is used in the texts. To keep the the terminology consistent the domain glosaries are usually used.

In contrast to the completely machine translation it is still the human translator who controls the whole process of translation. Nevertheless, the improvements in the machine translation allows to include the machine translation outputs to be included in the 

\section{Usual implementation}

... letter based editing distance, word based editing distance, 

... definition, traditional dynamic programming based algorithm, finite state machine can be constructed for Levensthein distance, for bigger databases a good index is necessary to find some candidate which are later 

\section{Current TM tools}

... most of professional translator use SDL Trados, some open-source projects, IBM released its own TM as open source, OmegaT widely used for localization of open source projects

... MyMemory project is somehow very similar to our project -- a big general purpose translation memory. 